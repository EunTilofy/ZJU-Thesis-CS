\section{研究计划进度安排及预期目标}

\subsection{进度安排}

\begin{enumerate}
    \item 2025.2.28 以前,阅读有关大语言模型偏好学习相关论文,确定毕业论文选题。
    \item 2025.3.15 以前,完成开题报告,文献综述和外文翻译,初步确定技术路线。
    \item 2025.3.31 以前,复现参考文献SPA的算法流程,对于自增强的偏好学习范式进行更深入的理解。
    \item 2025.4.15 以前,建模基于对抗性偏好学习的算法框架,并进行理论分析。
    \item 2025.4.30 以前,完成算法设计和代码编写,并在现有数据集上跑通实验。
    \item 2025.5.15 以前,完成更多补充实验,对算法进行更深入的评估。
    \item 2025.5.25 以前,完成毕业论文的撰写,准备毕业答辩。
\end{enumerate}

\subsection{预期目标}

\begin{enumerate}
	\item 设计高效的偏好优化算法:通过引入对抗性训练机制,结合数据扩展和自我优化策略,优化大语言模型的对齐度,使其在少量人工标注数据的支持下,通过多轮迭代逐步提升实际任务中的表现。
	\item 减少人工标注数据需求:利用自标注和偏好反馈生成机制,显著降低对大量人工标注偏好数据的依赖,使模型在数据稀缺的环境下仍能实现高效对齐,目标是在标注数据量减少至原始需求的1/30时,仍能取得较好性能。
	\item 提高模型的鲁棒性和适应性:通过对抗性训练和自我优化策略,确保模型能够在面对噪声数据时进行自我纠正,并维持高适应性,尤其在不确定性较大的应用场景中保持较高的对齐效果。
	\item 进行理论分析与性能评估:对所提出的算法进行系统的理论分析,确保其在面对数据噪声和偏差时具有鲁棒性,并进行大量实验验证其在不同任务和数据规模下的效果,确保算法具备广泛的实用性和推广价值。
	\item 探索数据扩展与自我优化的潜力:进一步探索数据扩展和自我优化策略,提升模型在各种任务中的泛化能力和适应能力,从而增强其在不同应用场景中的表现和效率。
\end{enumerate}