\section{项目的主要内容和技术路线}

\subsection{主要研究内容}

本研究旨在探索并证明在线带测试的多处理器调度问题的确定性算法竞争比下界。
即:证明不存在竞争比总是低于某个下界的确定性算法。

我将对问题进行程序建模,通过分析特定情况下的竞争比表现,识别出影响竞争比的关键因素。
在此基础上,我还将对现有算例进行分析,特别关注那些导致竞争比达到最坏的特殊情况。
最后,我将运用计算机辅助的方法在有限的时间内搜索出更广泛的实例,
以验证该问题的竞争比下界,并通过严谨的分析来确认最终结果。

\subsection{技术路线}

我将尝试沿用 Gormley 等设计的线性在线问题的竞争比下界求解模型~\cite{gormley2000generating},
将其应用到在线带测试的多处理器调度问题的竞争比下界的研究中。
依据该论文的思路,我将先建立传统的博弈搜索模型,
将调度的过程视为调度算法与对手的 $n$ 轮博弈过程。
每一轮博弈分为以下三步:
第一步对手指定当前任务的 $t_i$ 和 $u_i$;
第二步算法为当前输入的任务分配一个处理器并决定是否测试;
第三步若算法决定测试则对手还要指定当前任务的 $p_i$。
算法的最终目的是达到预设的竞争比 $\alpha$,
而对手的最终目的则是阻止算法达到该竞争比。
如果最终的博弈结果是对手获胜,
则证明无论如何决策,总存在一个实例使竞争比大于 $\alpha$,即竞争比下界大于 $\alpha$。

传统的博弈搜索模型将使用搜索剪枝的方法,暴力搜索所有可能的数据并剪去目前已不可能达到最优解的数据。
Gormley 的模型~\cite{gormley2000generating}则考虑到了调度问题的“线性性”,即博弈搜索树的构建可视为一个带max、min运算的线性规划模型,
进而通过拆分 $\text{max}$ 和 $\text{min}$ 运算得到多个传统线性规划模型。
从而一方面可以直接利用线性规划算法求解,
另一方面也可以通过线性规划的解的性质,将解的范围限制在有限的空间内,对对手的决策进行优化来辅助剪枝。

最后,我将用 C++ 语言实现上述最优化模型和搜索框架,先对 $m=1$ 和 $m=2$ 的具有较为成熟的情形验证框架的准确性,
再探索 $m=3, n\geq 4$ 更精确的竞争比下界。

\subsection{可行性分析}

现有的对竞争比下界的研究大多基于手工构造特殊数据。
这对 $m\geq 3, n\geq 4$ 的情况已经很难奏效。
因此借助计算机的强大算力构造人力难以搜索到的数据来提升竞争比下界是大有可为的。

此外,Gormley~\cite{gormley2000generating} 的论文中引用了一个定理:
\begin{thm}
    若在线问题是线性的,
    给定任意实数 $\alpha>1$ 和整数 $n$,存在一个算法,
    可在有限时间内构造出 $n$ 层的博弈树证明在线问题的竞争比下界不小于 $\alpha$,
    或证明竞争比下界大于 $\alpha$。
\end{thm}
这个定理保证了算法的可停机性。然而,由于直接拆解最优化问题会导致子问题的指数级增加,
我们仍需设计一些搜索剪枝的操作以提高程序上运行效率,
使其能在有限的时间内寻找到具有更大竞争比下界的实例。