{
    \setlength{\parindent}{0em}
    \par {\zihao{4}\bfseries 一、题目:\Title}
    \\
    \par {\zihao{4}\bfseries 二、指导教师对文献综述、开题报告、外文翻译的具体要求:}
}

围绕面向数据高效的大语言模型对齐任务的对抗性偏好优化算法展开研究。首先,需调研与该课题相关的最新技术和研究成果,特别是对抗性训练、偏好优化、数据扩展和自我优化等方面的文献。要求阅读10~15篇中英文文献,选择最具代表性的一篇外文文献完成文献翻译。深入理解当前大语言模型对齐的研究现状,并在此基础上完成3000字以上的文献综述。
文献综述需总结大语言模型对齐的基本思路与方法,结合个人理解,分析现有方法的优缺点。通过综述为研究目标和方法提供理论支持,并在此基础上撰写3500字以上的开题报告。报告应详细阐述该领域的基本理论、研究目标、方案及进度安排,展示对该领域的全面理解与研究规划。

项目的核心任务是设计一种创新的对抗性偏好优化算法,减少人工偏好数据的需求,提高大语言模型对齐任务的效率。研究过程中应通过文献学习和算法设计,培养严谨的研究作风,提升理论与实践结合的能力。最终,研究成果应为大语言模型对齐任务提供一种高效且数据需求较少的优化方案。

\mbox{} \vfill

\signature{指导教师(签名)}
% comment the line above and uncomment the line below if you want to set a signature with a specific date.
% \signaturewithdate{指导教师(签名)}{1897}{5}{21}
